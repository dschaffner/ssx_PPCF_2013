\documentclass[12pt]{article}

%\voffset=0.5in
\topmargin=-0.5in
%\footskip=0.0in
\textheight=9.0in
%\pagestyle{empty}		

\begin{document}

\begin{center}
{\bf \large MHD turbulence statistics for \\ 
decaying double-helix flux rope plasmas \\
D. Schaffner, A. Wan, V. Lukin, M. Brown \\
Swarthmore College \\
PPCF special issue on flux ropes}
\end{center}

{\bf Abstract:} We have previously generated elongated Taylor double-helix flux rope plasmas in the SSX MHD wind tunnel.  These plasmas are remarkable in their rapid relaxation (about one Alfv\'en time) and their description by simple analytical Taylor force-free theory despite their high plasma $\beta$ and high internal flow speeds.  We discuss here the possibility that the turbulence facilitating access to the final state supports coherent structures and intermittency revealed by non-Gaussian signatures in the statistics.  Comparisons to a two-fluid simulation show a similarity in several statistical measures.

\bigskip {\bf Overview} Flux ropes observed in the heliosphere have two striking properties.   First is their rapid emergence.  Whether in the magnetosphere or in the solar corona, these large scale structures emerge rapidly, often in just a few Alfv\'en crossing times of the system.  Second is their long lifetimes.  Once formed, these structures persist for long times despite being embedded in turbulent MHD plasma.  

Flux ropes have recently been observed {\it in situ} at the subsolar magnetopause \cite{Oieroset11}.  In this remarkable coordinated observation using three THEMIS spacecraft, a flux rope is caught in the process of forming, revealing properties that are fundamentally 3D. Since magnetospheric flux ropes evolve rapidly, observations of flux ropes in the magnetosphere tend to be detected at later stages of their evolution.  The newly formed flux rope reported here appeared to be flanked by two active X-lines as part of the formation process.

Flux ropes are also observed remotely in the solar atmosphere \cite{Patsourakos13}.  On July 19, 2012, an eruption occurred on the solar surface producing dynamical magnetic activity resulting in a destabilized flux rope, the acceleration of a fast ($1000~km/s$) coronal mass ejection, and a long-lived solar flux loop.  The long-lived structure is remarkable in its nearly semi-circular topology, and the persistence of a ``coronal rain'' from the loop top for nearly 24 hours.  The observation was made with the Solar Dynamics Observatory's AIA instrument on the sun's lower right hand limb.   This represents the first direct evidence of a fast CME driven by the prior formation and destabilization of a coronal magnetic flux rope.

Video at http://www.youtube.com/watch?v=HFT7ATLQQx8

It is interesting to note that in MHD simulations, the peak in the mean square current density $\langle j^2 \rangle$ is also achieved rapidly, within a fraction of an Alfv\'en time. At this time, the turbulence is fully developed, the peak of small scale activity is achieved, and coherent structures appear. 

\bigskip {\bf Review of prior result:} We have recently reported on the observation of a long-lived helical flux rope called a Taylor double-helix in the SSX MHD wind tunnel \cite{Gray13}.  The Taylor double-helix is the natural relaxed state of MHD plasma confined in a long, perfectly conducting cylinder \cite{Taylor86}.  In the case of an infinite cylinder, the minimum energy state has a helical pitch of $ ka = 1.234$, where k is the wave number associated with the z axis.  

In the SSX experiments, a magnetized plasma gun launches a magnetized plasma plume into a long flux conserving cylinder.  The plasma rapidly relaxes to the double-helix state in about 1 Alfv\'en crossing time and subsequently decays resistively.  In the paper, we postulated that the physics of selective decay was at play as the initially turbulent plasma relaxed to the double-helix state.  The selective decay hypothesis posits that the energy selectively decays relative to the magnetic helicity because the energy spectra peaks at higher wave numbers, where dissipation is higher \cite{Matthaeus80}.  The wind tunnel's minimum energy state possesses $ka = 1.292$, which is within 5\% of the infinite cylinder's $ka = 1.234$. 

Servidio et al. \cite{Servidio08,Servidio11} detail simulations which observe the rapid and simultaneous magnetohydrodynamic relaxation into localized patches of plasma with near alignment of {\bf J} and {\bf B}. These patches of locally relaxed plasma can then negotiate with adjacent patches to reach a globally relaxed state on a longer time scale.  However, many of the characteristics of the relaxed state will be evident locally. This localized relaxation might explain the rapidity of the transition observed in the double-helix plasmas.  A fully relaxed Taylor state would be expected to have a flat lambda profile (where $\nabla \times B = \lambda B$ governs the equilibrium).  The reported lack of a flat radial lambda profile could also be a consequence of a patchy relaxation.

We suspect that the MHD turbulent flow in the SSX wind tunnel contains patches of locally relaxed plasma with reconnection sites at the boundaries.  A fully relaxed flow might be expected to exhibit Gaussian statistics in its fluctuations and power law behavior for the power spectra.  A flow containing coherent structures and reconnection sites should exhibit non-Gaussian statistics.  Simulations show that coherent structures appear rapidly, in less than one dynamical time.  Large numbers of reconnection sites can be identified statistically in MHD turbulence studies \cite{Servidio09,Servidio10a}.  A statistical way to find these coherent structures is to identify rapid changes in the magnetic field vector.  A useful technique is to generate a probability distribution function PDF of vector increments \cite{Greco08,Greco09}.

Non-Gaussian statistics and characteristic coherent structures are initiated almost identically in dissipative and ideal systems \cite{Wan09}. Therefore we postulate that the origins of coherence and intermittency are essentially ideal, with dissipation acting only to limit growth of the smallest scale structures.  The fact that our Lundquist number is modest (S = 1000) shouldn't impact the emergence of coherent structures.

We are interested here in the decay phase of the double-helix, in particular, processes that rapidly evolve the state such as patchiness and the evolution of coherent structures.  We present MHD turbulence statistics that suggest the emergence of non-Gaussian structures.

I think the theme of the paper should be that we see this state \cite{Gray13}, and in the PRL we hypothesized that there might be rapidly formed patches of relaxation.  These can be exposed by statistical studies, esp departures from Gaussian statistics.   Connection to HiFi could be some direct comparisons of turbulent statistics.
 
 \bigskip {\bf Single plume statistics:} 
 
 Figures: 1. SSX schematic/Taylor state.  2. Time series, mean values of B, n, T.  3. Fluctuation spectra $E_B(\omega), E_V(\omega)$.  4. Radial correlation function.  5. PDF of increments.  6. a nice HiFi twisted flux rope (rendered with VIsIt). 7. comparison with HiFi data (spectrum, PDF).
 
 Description of the experimental set up (Figs 1 and 2). Especially showing the decay phase.  Similar setup as described in \cite{Gray13}.  The same long cylindrical copper flux conserver, 7.8 cm in radius, is used but fitted with guns at either end giving it an aspect ratio of $L/a \cong 11$.  In these experiments, we operate with more stored energy in the capacitor banks (up to 8 kJ).  The gun injects a plasma plume into a highly evacuated flux conserving wind tunnel at $50~km/s$.
 
Description of statistical tools.  Wavelet for spectra.  The idea that a PDF of  $\Delta b(t, \Delta t) = b(t + \Delta t) - b(t)$ accentuates intermittent events.

Greco, {\it et al} \cite{Greco08} identified intermittent structures in MHD turbulence simulations using statistical techniques then connected the structures with regions of high current density.  Using data from a high resolution 3D MHD simulation, they constructed PDFs of magnetic field vector increments $\Delta {\bf B} = |{\bf B}(s + \Delta s) - {\bf B}(s)|$ for two scale separations $\Delta s$; one much smaller than the correlation scale and another about 2 $\lambda_C$.  They found that the PDF of $\Delta {\bf B}$ at the larger increment was close to Gaussian indicating that increments larger than the correlation scale are normally distributed.  However, the PDF at the smaller increment had substantial tails in the distribution and furthermore, had the same distribution as the PDF of a component of the current density, indicating that the non-Gaussian statistics were correlated with intense current sheets.  They also compared the identification of  discontinuities using the spatial interval method described above with coherent structures identified using (PVI) intermittency statistics and found the techniques to be similar.

In a follow-up investigation, Greco, {\it et al} \cite{Greco09} studied the statistics of ACE solar wind data as well as 2D and 3D simulation data using the same techniques.  Time series were analyzed for the solar wind data while spatial separations were analyzed from the simulations.  First, they found that the ACE solar wind data had nearly identical increment PDFs to the 2D simulation data.  Second, they were able to correlate specific features in the 2D simulation to departures from Gaussian statistics in the PDF.  A narrow inner peak is super-Gaussian and corresponds to low values of fluctuations in the lanes between magnetic islands.  An intermediate range is sub-Gaussian and corresponds to fluctuations in current cores inside magnetic flux tubes.  Finally, at several standard deviations, there are super-Gaussian wings corresponding to coherent small-scale current sheet-like structures that form the sharp boundaries between the magnetic flux tubes.
 
 Description of HiFi as implemented for these runs.  Dissipation model.  Hall.  Recycling choice.  
 
\bigskip {\bf Some quotes and ideas from Servidio, Wan, Greco, Matthaeus:} 
 
 In this complex scenario, reconnection is spontaneous but locally driven by the fields, with the boundary conditions provided by the turbulence.

These results explain how rapid reconnection occurs in MHD turbulence in association with the most intermittent non-Gaussian current structures, and also how turbulence generates a very large number of reconnection sites that have very small rates.

Very recent results in this line of study, involving reconnection rates for turbulence at varying times, and the first look at how Hall effect influences reconnection in turbulence, have also been highlighted here. From the freely decaying turbulence, time dependent study, we have found that the reconnection rate distribution evolves rapidly from a state that has essentially no fast reconnection sites, and develops a hard distribution that has a highly enhanced tail of strong rates, in a time of the order of the peak turbulence dissipation time scale. Subsequently, as the turbulence ages and begins to slow down, so also does the reconnection rate distribution soften, with the tail of strong rates diminishing in just a few nonlinear times.

The coherency of a turbulent pattern is hidden in the phases of the Fourier expansion.

The production of these spatial patches of correlations and anticorrelations requires that the statistical distribution of velocity and magnetic field become non-Gaussian, in particular, because each type of relaxation requires that at least the fourth order correlations become non-Gaussian. Our conclusion is that this multifaceted rapid relaxation is intimately related to the formation of spatial intermittent structures.

Non-Gaussian statistics and characteristic coherent structures are initiated almost identically in dissipative and ideal systems. Therefore we postulate that the origins of coherence and intermittency are essentially ideal, with dissipation acting only to limit growth of the smallest scale structures.

An alternative hypothesis examined here is that observed discontinuities are small scale coherent structures that emerge in fully developed intermittent MHD turbulence.

These coherent structures can appear in less than one nonlinear time beginning from Gaussian initial data that completely lack coherent structures.

A way to find regions of high magnetic stress and coherent structures is to identify rapid changes in the magnetic field vector.

The results detailed here suggest distinctive kinetic signatures observed in solar wind plasma are associated statistically with coherent magnetic structures such as current sheets, which are connected to the intermittency properties of MHD turbulence. 

\begin{thebibliography}{99}

\bibitem{Oieroset11}
Oieroset, M., T. D. Phan, J. P. Eastwood, M. Fujimoto, W. Daughton, M. Shay, V. Angelopoulos, F. S. Mozer, J. P. McFadden, D. E. Larson, and K. -H. Glassmeier, Direct evidence for a three-dimensional magnetic flux rope flanked by two active magnetic reconnection X-lines at the Earth's magnetopause, Physical Review Letters, Vol. 107, 165007, 2011

\bibitem{Patsourakos13}
S. Patsourakos, A. Vourlidas, and G. Stenborg. Direct Evidence for a Fast Coronal Mass Ejection Driven by the Prior Formation and Subsequent Destabilization of a Magnetic Flux Rope,
ApJ 764 125, 2013

\bibitem{Gray13} 
T. Gray, M. R. Brown, and D. Dandurand, Observation of a Relaxed Plasma State in a Quasi-Infinite Cylinder, Phys. Rev. Letters 110, 085002 (2013). 

\bibitem{Taylor86} J. B. Taylor, Rev. Mod. Phys. 58, 741 (1986).

\bibitem{Matthaeus80} W.H. Matthaeus and D. Montgomery, Ann. N.Y. Acad. Sci. 357, 203 (1980).

\bibitem{Servidio08}
Servidio, S., Matthaeus, W. H., and Dmitruk, P.: Depression of Nonlinearity in Decaying Isotropic MHD Turbulence, Phys. Rev. Lett., 100, 095005, doi:10.1103 Phys. Rev. Lett.100.095005, 2008.

\bibitem{Servidio11}
Servidio, S., Dmitruk, P., Greco, A., Wan, M., Donato, S., Cassak, P. A., Shay, M. A., Carbone, V., Matthaeus, W. H., Magnetic reconnection as an element of turbulence, Nonlinear Processes in Geophysics, Vol. 18, p. 675-695, 2011. 

\bibitem{Servidio09}
Servidio, S., Matthaeus, W. H., Shay, M. A., Cassak, P. A., and Dmitruk, P.: Magnetic Reconnection in Two-Dimensional Magnetohydrodynamic Turbulence, Phys. Rev. Lett., 102, 115003, doi:10.1103/PhysRevLett.102.115003, 2009.

\bibitem{Servidio10a}
Servidio, S., Matthaeus, W. H., Shay, M. A., Dmitruk, P., Cassak, P. A., and Wan, M.: Statistics of magnetic reconnection in two- dimensional magnetohydrodynamic turbulence, Phys. Plasmas, 17, 032315, doi:10.1063/1.3368798, 2010a.

\bibitem{Servidio10b}
Servidio, S., Wan, M., Matthaeus, W. H., and Carbone, V.: Local relaxation and maximum entropy in two-dimensional turbulence: Phys. Fluids, 22, 125107, doi:10.1063/1.3526760, 2010b.

\bibitem{Servidio11b}
Servidio, S., Greco, A., Matthaeus, W. H., Osman, K. T., and Dmitruk, P.: Statistical association of discontinuities and reconnection in magnetohydrodynamic turbulence, J. Geophys. Res., 116, A09102, 1�11, doi:10.1029/2011JA016569, 2011.

\bibitem{Greco08}
Greco, A., Chuychai, P., Matthaeus, W. H., Servidio, S., and Dmitruk, P.: Intermittent MHD structures and classical discontinuities, Geophys. Res. Lett., 35, L19111, doi:10.1029/2008GL035454, 2008.

\bibitem{Greco09}
Greco, A., Matthaeus, W. H., Servidio, S., Chuychai, P., and Dmitruk, P.: Statistical Analysis of Discontinuities in Solar Wind ACE Data and Comparison with Intermittent MHD Turbulence, Astrophys. J., 691, L111, doi:10.1088/0004-637X/691/2/L111, 2009.

\bibitem{Wan09}
Wan, M., Oughton, S., Servidio, S., and Matthaeus, W. H.: Generation of non-Gaussian statistics and coherent structures in ideal magnetohydrodynamics, Phys. Plasmas, 16, 080703, doi:10.1063/1.3206949, 2009.

\bibitem{Wan12}
Wan, M., W. H. Matthaeus, H. Karimabadi, V. Roytershteyn, M. Shay, P. Wu, W. Daughton, B. Loring, and S. C. Chapman, Intermittent Dissipation at Kinetic Scales in Collisionless Plasma Turbulence, Physical Review Letters, Vol. 109, 195001, 2012. 

\bibitem{Osman11}
Osman, K. T., Matthaeus, W. H., Greco, A., and Servidio, S.: Evidence for Inhomogeneous Heating in the Solar Wind, Astrophys. J., 727, L11, doi:10.1088/2041-8205/727/1/L11, 2011.


\end{thebibliography}

\end{document}